\documentclass{dai-working-paper}

%% ---- Metadata ----
\dai{2026}{001}
\programme{Consent Mechanics}
\doidoi{10.5281/zenodo.00000000}
\correspondingauthor{author@dissensus.ai}
\orcid{0000-0000-0000-0000}
\keywords{friction dynamics, multi-agent coordination, consent mechanics, delegation cost}
\jelcodes{C72, D82, O33}

\title{Example Working Paper: Friction Dynamics in Agent Coordination}
\author{Author Name\\{\normalsize Dissensus AI}}
\date{February 2026}

\begin{document}
\maketitle

\begin{abstract}
This paper investigates friction dynamics arising from delegation and coordination
in multi-agent systems. We formalise a decomposition of coordination overhead into
alignment, stake, and entropy components, and derive conditions under which
delegation produces net welfare gains despite non-trivial friction costs. The
framework is applied to both computational and institutional settings.
\end{abstract}

\section{Introduction}

Coordination among autonomous agents---whether human, institutional, or
computational---incurs overhead that existing game-theoretic models treat as
exogenous \citep{Ostrom1990}. This paper argues that coordination friction is
endogenous to the delegation structure itself and admits a tractable
decomposition.

Following \citet{Williamson1985}, we distinguish between alignment friction
(preference divergence between principal and agent) and entropic friction
(irreducible uncertainty in the coordination channel). Our contribution is a
formal bridge between these transaction-cost concepts and contemporary
multi-agent system design.

\section{Friction Decomposition}

\begin{definition}[Coordination Friction]
Let $\mathcal{A} = \{a_1, \ldots, a_n\}$ be a set of agents and $\delta$
a delegation mapping. The \emph{total friction} of $\delta$ is
\begin{equation}\label{eq:friction}
  \Phi(\delta) \;=\; \underbrace{\alpha(\delta)}_{\text{alignment}}
    \;+\; \underbrace{\sigma(\delta)}_{\text{stake}}
    \;+\; \underbrace{\eta(\delta)}_{\text{entropy}}.
\end{equation}
\end{definition}

\begin{proposition}\label{prop:bound}
Under Assumptions~\ref{asm:bounded}--\ref{asm:mono}, the total friction
$\Phi(\delta)$ is bounded above by
$\Phi(\delta) \leq \kappa \cdot \log|\mathcal{A}|$
for a constant $\kappa$ depending only on the entropy of the delegation channel.
\end{proposition}

\begin{proof}
The alignment component satisfies $\alpha(\delta) \leq H(\delta)$ by
construction. Combining with the monotonicity of $\sigma$ (Assumption~\ref{asm:mono})
yields the logarithmic bound. Full derivation in Appendix~A.
\end{proof}

\begin{assumption}\label{asm:bounded}
Agent preferences are bounded: $|u_i| \leq M$ for all $i$.
\end{assumption}

\begin{assumption}\label{asm:mono}
Stake friction $\sigma$ is monotonically non-decreasing in delegation depth.
\end{assumption}

\section{Discussion}

Equation~\eqref{eq:friction} provides an operational decomposition applicable to
both institutional governance \citep{Ostrom1990} and computational multi-agent
systems. The logarithmic bound in Proposition~\ref{prop:bound} suggests that
friction scales favourably with agent count, provided entropy remains controlled.

\section*{Acknowledgements}

The authors thank the Dissensus AI research group for comments on earlier drafts.

\begin{thebibliography}{9}
\bibitem[Ostrom(1990)]{Ostrom1990}
Ostrom, E. (1990).
\textit{Governing the Commons: The Evolution of Institutions for Collective Action}.
Cambridge University Press.

\bibitem[Williamson(1985)]{Williamson1985}
Williamson, O.~E. (1985).
\textit{The Economic Institutions of Capitalism}.
Free Press, New York.
\end{thebibliography}

\end{document}
